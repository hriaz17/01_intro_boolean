\documentclass [11pt]{article}
\usepackage{ua}

\begin{document}

\headerHW{1b}{Intro, Boolean Retrieval}{February 8, 2026}

\begin{instructions}
\begin{itemize}
\item Push your code on GitHub and modify the README file with your name so we can map your GitHub user name to your official name in d2l.

\item Steps to get started:

\begin{small}
  \begin{verbatim}
   # Accept the invite link for the assignment.
   #   Doing so will create a repo named ASSIGNMENT_NAME-YOUR-GITHUB_USER
   #
   # First, include a README file with your name so we can map your GitHub user name
   #   to your official name in d2l
   #
   # Then, set up the environment and get started:
   > git clone YOUR_REPO
   > cd YOUR_REPO
   > python3 -m venv .env
   > source .env/bin/activate
   > pip install -r requirements.txt   
  \end{verbatim}
\end{small}

\item You can play with the search engine using the following command: \texttt{python3 boolean\_engine.py wiki-small.txt}. For example:

\begin{small}
  \begin{verbatim}
  > python3 boolean_engine.py wiki-small.txt
  Query: algorithm
  [1552, 1948, 3035, 5033, 5243]
  Query: NOT the
  [1, 2, 5, 7, 12, 14, 16, 18, 19, 24, (and many more)]
  Query: AND algorithm the
  [1552, 1948, 3035, 5033, 5243]
  \end{verbatim}
\end{small}
  
Note that the syntax is ``operator first-term second-term'', and that operators must be in upper case (e.g., ``and'' is considered an term, not an operator.

\item Grading will be primarily via test cases.
You have access to a few public test cases.
You can run the public test cases locally on your computer with the following command:
\texttt{> pytest}.
For details about which test cases pass and fail, use \texttt{> pytest --verbose}.

\textbf{In addition to running the public test cases, we will check your implementation with a set of hidden test cases that are not visible to you, to ensure that your implementation is generalizable}.
You will not receive credit if you hard code the expected output or use external libraries that are not in \texttt{requirements.txt}.
Additionally, you must implement what you are asked using the methods you are asked (or as discussed in class).
For example, you will not get credit if you do not use an inverted index.

The public test cases will also run after you push your code to GitHub and create a pull request (see repository for details).
You are responsible for ensuring that the public test cases run on GitHub.
The instructor will test your code with the hidden test cases after the submission deadline has passed.
See details including screenshots in the repository.

\item HW\#1b is worth 30 points of your final grade but graded out of 100 points (undergraduate students) or 150 points (graduate students).
  \begin{itemize}
  \item Undergraduate students do not need to submit the questions marked as ``graduate students,'' but must submit all other questions (or be ok loosing points).
If they submit answers to questions marked as ``graduate students,'' their grade will be capped at 100  points. See HW\#1a for an example. 

  \item Graduate students must do questions marked as ``graduate students'' or be ok loosing points for their final grade. See HW\#1a for an example.
  \end{itemize}
\end{itemize}

\end{instructions}
\newpage

\begin{question}{5}
Browse the code provided to you.
Specifically:
  \begin{itemize}
  \item The \texttt{\_\_init\_\_(self, f)} method in \texttt{class IRSystem} creates the inverted index
  \item Two operators are already implemented for you:
    \texttt{q\_term} and \texttt{q\_not}
  \item \texttt{test\_boolean\_queries.py} contains the test cases.
  \end{itemize}
\end{question}

\begin{question}{45}
Implement the \texttt{q\_and} method.
Your implementation is most likely correct when you pass the public test case (but for full points you also need to pass the hidden test cases).
\end{question}

\begin{question}{50}
Implement the \texttt{q\_or} method.
Your implementation is most likely correct when you pass the public test case (but for full points you also need to pass the hidden test cases).
\end{question}

\begin{question}{50}
\textbf{This question is for graduate students.}

Modify the \texttt{run\_query(self, query):} method to handle arbitrary boolean queries.
This is substantially harder than the other two questions.
Here is an example from the public test case, but your implementation should handle any boolean query, no matter how long.

  \begin{verbatim}
  AND president OR house white
  Result: [1325, 2391, 3041, 3618, 3771, 4053, 5409]
  \end{verbatim}
  
  Your implementation is most likely correct when you pass the public test case (but for full points you also need to pass the hidden test cases).
  Note that the syntax may look weird at first.
  However, it will substantially ease the (potential) pain involved in implementing a parser with parenthesis.
  You only need a stack!

\end{question}

\end{document}
